\documentclass{article}
\usepackage[inner=2.5cm,outer=2.5cm,top=4.5cm,bottom=4.5cm]{geometry}
%\documentclass[]{tufte-book} % Use the tufte-book class which in turn uses the tufte-common class
\pdfoutput=1
\usepackage[super,numbers]{natbib}
\usepackage{aas_macros}

\usepackage{microtype} % Improves character and word spacing

\usepackage{lipsum} % Inserts dummy text

\usepackage{booktabs} % Better horizontal rules in tables
\usepackage{epsdice}
\usepackage{url}
\usepackage[svgnames]{xcolor}
\usepackage[colorlinks,backref=page]{hyperref}
\hypersetup{
    colorlinks = true,
    citecolor = blue,
    linkcolor = blue,
    urlcolor = CornflowerBlue,%svgname when using xcolor
}
\urlstyle{same}

\usepackage{graphicx} % Needed to insert images into the document
\graphicspath{{graphics/}} % Sets the default location of pictures
\setkeys{Gin}{width=\linewidth,totalheight=\textheight,keepaspectratio} % Improves figure scaling

\usepackage{fancyvrb} % Allows customization of verbatim environments
\fvset{fontsize=\normalsize} % The font size of all verbatim text can be changed here

\newcommand{\hangp}[1]{\makebox[0pt][r]{(}#1\makebox[0pt][l]{)}} % New command to create parentheses around text in tables which take up no horizontal space - this improves column spacing
\newcommand{\hangstar}{\makebox[0pt][l]{*}} % New command to create asterisks in tables which take up no horizontal space - this improves column spacing

\usepackage{xspace} % Used for printing a trailing space better than using a tilde (~) using the \xspace command
\newcommand{\veco}{\ensuremath{\mathbf{\Omega}}\xspace}
\newcommand{\vecv}{\mathbf{v}\xspace}
\newcommand{\msun}{\ensuremath{\mathrm{M}_\odot}\xspace}
\newcommand{\lsun}{\ensuremath{\mathrm{L}_\odot}\xspace}
\newcommand{\ehat}{\hat{\mathbf{e}}\xspace}
\usepackage{amsmath}
\usepackage{mathtools}


\begin{document}

\noindent\textbf{Problem Set 1: Molecular Clouds}

\begin{enumerate}
    \item \textbf{Cloud Lifetimes}\\
``Typical'' molecular clouds have mean densities of $\sim100$ H$_2$ molecules per
cubic centimeter and span scales $\sim10$ pc.


Molecular clouds have temperature $T\sim20$ K.
The sound speed in molecular gas is $c_s = \sqrt{\frac{k_B T}{m}}$, where $m$ is the
mass of the particle, $k_B$ is the Boltzmann constant, and $T$ is the
temperature in Kelvin.

\begin{enumerate}
    \item The mass fraction of hydrogen, helium, and metals is\citep{Kauffmann2008}
        \begin{itemize}
            \item $X(H)=\frac{M(H)}{\mathcal{M}}=\frac{\mu_H N(H)}{\mathcal{M}}=0.71$
            \item $Y(He)=\frac{M(He)}{\mathcal{M}}=0.27$
            \item $Z(metals)=\frac{M(Z)}{\mathcal{M}}=0.02$
        \end{itemize}
        where $\mathcal{M}$ is the total mass, $\mu_H$ is the mass of a
        hydrogen atom, $N(H)$ is the number of hydrogen atoms, and $M(...)$ is the mass of the specified particle type.
        Show that the mean molecular mass, i.e., the mean mass per H$_2$
        particle, is 2.82 AMU, and the mean mass per free particle is 2.36 AMU.
        You may assume any reasonable value for $M(Z)$.
    \item Compute: What is the free-fall timescale in molecular clouds?  
        Recall, as shown in Krumholz \S6.3,
        $$t_{ff} = \sqrt{\frac{3 \pi}{32 G \rho}}$$
    \item Compute: What is the crossing time of a typical molecular cloud?  The
        crossing time is the time required for a sound wave to cross the cloud.
    \item How do these compare?
    \item If the molecular cloud forms stars at 100\% efficiency (all of the
        gas becomes stars at some point), and it collapses in one free-fall
        time, what is the star formation rate?
        Given that the Galactic molecular gas mass is $\sim10^9$ \msun, what is the MW SFR?
        Do we expect stars to form at this rate?
\end{enumerate}

\item  \textbf{Observing Clouds} \\
    Molecular clouds are primarily comprised of hydrogen molecules, H$_2$.
    Answer the following questions about the Williams Ch 7 handout
    and Ch.1 of Krumholz.

    \begin{enumerate}
        \item How do we observe molecular clouds?
            Why do we not directly observe the H$_2$ molecule?
        \item What molecules are commonly observed in the ISM?
            For each molecule, note whether it is particularly important for any specific type
            of measurement (e.g., does it trace `high'- or `low'- density gas?  Can it be
            used to measure temperature?).\footnote{
            You should not provide an exhaustive list, just note the molecules discussed in the
            assigned chapters.  Fully exhaustive lists can be found at
            \url{http://www.astrochymist.org/astrochymist_ism.html} and
            \url{https://ui.adsabs.harvard.edu/abs/2022ApJS..259...30M/abstract}.}
    \end{enumerate}

%\item \textbf{Stars and Gas} \\
%    Why do galaxies appear roughly the same when you look at the gas vs. when you look at the stars?

\end{enumerate}


\bibliographystyle{aasjournal}
\bibliography{bibliography}
\end{document}


