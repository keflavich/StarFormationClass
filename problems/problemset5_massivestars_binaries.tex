
These are ideas of problems from Tom Megeath's class that could reasonably be expanded into problems for undergrads

2. Determine the Eddington luminosity for each mass and draw it over the
tracks. Using the “classical” Eddington luminosity which uses the Thompson
scattering cross section. How does the maximum luminosity of the tracks compare
to the Eddington luminosity?

Draw on Figure 3 of https://ui.adsabs.harvard.edu/abs/1990A%26AS...84..139M/abstract


What mass infall rate is needed is needed sustain constant Deuterium burning in
the center of a protostar? Assume a standard abundance and use a protostar mass
of 0.5 Msun and a radius of 2 Rsun. Also assume the protostar is convective so
that Deuterium landing on the surface of the star is transported to the center
of the star.




1. We showed that in a flow of infalling gas, where the gas is in free fall toward the central protostar, the continuity equation (i.e. conservation of mass) requires that the gas density has the following dependency on radius:
ρ ∝ r−3/2 (1) Now consider a constant velocity flow where there is no acceleration (either infall or outflow,
the direction of the flow is irrelevant). The density is given by:
ρ ∝ r−α (2)
Using the equation of continuity, what is α?



1. Below are the six hyperfine components of the J = 1 -> 0 transition of the molecular ion N2H+ [I have included more information on N2H+ at the end of the homework if you are curious]. This transition is shown for four dense cores (Caselli, Bense, Myers & Tafalla, 2002). The spectra show intensity (given as Antenna Temperature in units of K) vs frequency. The frequency has been converted into velocity using the Doppler shift formula and the rest frequency of the N2H+ (1 -> 0) components.
a. Measure the internal 1-D velocity dispersion by using the leftmost component (F1F = 01 -> 12, see last page), which is isolated and not blended with a nearby hyperfine component. First, measure the Full Width at Half Maximum (FWHM) velocity for that component (just the width of the line at half the maximum intensity level). Do this for all 6 cores. You will need to do this by counting bins or using a ruler.
b. Assuming the lineshape is a Gaussian function, convert the FWHM into the velocity dispersion (i.e. the standard deviation of the Gaussian). You need to first derive a simple relationship between the FWHM and the σ of the Gaussian.
c. Now adopt a kinetic temperature of 10 K for all cores. Assume the non- thermal and thermal velocities have a Gaussian shaped distribution, giving σtot2 = σNT2 + σTH2 where σtot is the total velocity distribution, σNT is the non- thermal velocity distribution, and σTH is the thermal velocity distribution. (the non-thermal velocity is a combination of all motions not generated by the thermal motions of the individual molecules: rotation, infall, outflow and turbulence). Using your derivation of σTH vs temperature from the last HW, calculate σNT and σNT/ σTH for all 6 cores.




PROBLEM SET 5 : OBSERVATIONS


1. N2H+ line above from Williams
2. Thermal emission from a blackbody of a given size
