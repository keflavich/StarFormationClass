%\documentclass[twoside]{tufte-book} % Use the tufte-book class which in turn uses the tufte-common class
\documentclass{article}
\pdfoutput=1
\usepackage{url}
\usepackage{graphicx}
\usepackage{natbib}
\usepackage{xspace}

\usepackage{microtype} % Improves character and word spacing

\usepackage{lipsum} % Inserts dummy text

\usepackage{booktabs} % Better horizontal rules in tables
\usepackage{epsdice}

\usepackage{graphicx} % Needed to insert images into the document
\graphicspath{{graphics/}} % Sets the default location of pictures
\setkeys{Gin}{width=\linewidth,totalheight=\textheight,keepaspectratio} % Improves figure scaling

\usepackage{fancyvrb} % Allows customization of verbatim environments
\fvset{fontsize=\normalsize} % The font size of all verbatim text can be changed here

\newcommand{\hangp}[1]{\makebox[0pt][r]{(}#1\makebox[0pt][l]{)}} % New command to create parentheses around text in tables which take up no horizontal space - this improves column spacing
\newcommand{\hangstar}{\makebox[0pt][l]{*}} % New command to create asterisks in tables which take up no horizontal space - this improves column spacing

\usepackage{xspace} % Used for printing a trailing space better than using a tilde (~) using the \xspace command
\newcommand{\veco}{\ensuremath{\mathbf{\Omega}}\xspace}
\newcommand{\vecv}{\mathbf{v}\xspace}
\newcommand{\msun}{\ensuremath{\mathrm{M}_\odot}\xspace}
\newcommand{\rsun}{\ensuremath{\mathrm{R}_\odot}\xspace}
\newcommand{\lsun}{\ensuremath{\mathrm{L}_\odot}\xspace}
\newcommand{\ehat}{\hat{\mathbf{e}}\xspace}
\usepackage{amsmath}
\usepackage{mathtools}

% \usepackage{natbib}
% 
% \usepackage{microtype} % Improves character and word spacing
% 
% \usepackage{lipsum} % Inserts dummy text
% 
% \usepackage{booktabs} % Better horizontal rules in tables
% 
% \usepackage{graphicx} % Needed to insert images into the document
% \graphicspath{{graphics/}} % Sets the default location of pictures
% \setkeys{Gin}{width=\linewidth,totalheight=\textheight,keepaspectratio} % Improves figure scaling
% 
% \usepackage{fancyvrb} % Allows customization of verbatim environments
% \fvset{fontsize=\normalsize} % The font size of all verbatim text can be changed here
% 
% \newcommand{\hangp}[1]{\makebox[0pt][r]{(}#1\makebox[0pt][l]{)}} % New command to create parentheses around text in tables which take up no horizontal space - this improves column spacing
% \newcommand{\hangstar}{\makebox[0pt][l]{*}} % New command to create asterisks in tables which take up no horizontal space - this improves column spacing

\usepackage{xspace} % Used for printing a trailing space better than using a tilde (~) using the \xspace command

\begin{document}

\noindent\textbf{Problem Set: Protostars}


\begin{enumerate}


    \item Phases of Protostellar evolution

        \begin{enumerate}
            \item What phases does a protostar go through during collapse from a prestellar core through the first ignition of nuclear burning?

                What physical processes are involved?

            \item How does the core temperature evolve during this time?

                Label this
                plot of evolution, $T_c$ vs $t$, from the start of
                collapse until the start of the main sequence. We assume the
                cloud started with a temperature of 10K.

                Label the Y-axis with the appropriate temperature (to an order
                of magnitude) and label each segment with the physical
                processes that are going on during that time period.

                \begin{figure}[h!]
                \includegraphics{TimeEvolutionPlot.pdf}
                \end{figure}
                


        \end{enumerate}
\item {\bf A Simple Protostellar Evolution Model.}\\
Consider a protostar forming with a constant accretion rate $\dot{M}$. The
accreting gas is fully molecular, arrives at free-fall, and radiates away a
luminosity $L_{\rm acc} = f_{\rm acc} G M \dot{M}/R$ at the accretion shock,
where $M$ and $R$ are the instantaneous protostellar mass and radius, and
$f_{\rm acc}$ is a numerical constant of order unity. At the end of contraction,
the resulting star is fully ionized, all its deuterium has been burned to
hydrogen, and it is in hydrostatic equilibrium. The ionization potential of
hydrogen is $\psi_I = 13.6$ eV per amu, the dissociation potential of molecular
hydrogen is $\psi_M=2.2$ eV per amu, and the energy released by deuterium
burning is $\psi_D\approx 100$ eV per amu of total gas (not per amu of
deuterium).

 We will first consider a low-mass protostar whose internal structure is
 well-described by an $n=3/2$ polytrope.


\begin{enumerate}
    \item 
 Since it is in equilibrium, the gravitational and thermal energy
 must be set by the Virial theorem, $|E_{grav}| =2 |E_{thermal}|$.
 For a polytrope, $E_{grav} = \frac{3}{5-n} \frac{G M^2}{R}$.
 Write down an expression for the total energy of the star,
 including thermal energy, gravitational energy, and the chemical energies
 associated with ionization, dissociation, and deuterium burning.
\item Use your expression for the total energy to derive an evolution equation
    for the radius for a star.  i.e., obtain an equation for $\dot{R}=\frac{dR}{dt}$.
    % Start by taking the derivative of your energy equation.
    %Assume that the change in energy with respect to time equals the luminosity, $L=|\dot{E}|$.
\item Once you have this equation for $\dot{R}$, rearrange it to obtain an equation in terms
    of $\frac{dR}{dM}$ ($\frac{dR}{dt} / \frac{dM}{dt} = \frac{dR}{dM}$).
\item
    \label{item:dlnrdlnm}
    Rearrange this equation further to obtain $\frac{d \ln R}{d \ln M}$,
    recalling that $d \ln x = \frac{1}{x}dx$.
\item
    \label{item:luminosity}
    Write down an equation for the star's luminosity, which includes both the accretion 
    luminosity and the nuclear burning luminosity. 
    To obtain the nuclear burning luminosity,  assume the star is always on the Hayashi track,
    which for the purposes of this problem we will approximate as having a
    fixed effective temperature $T_{\rm H} = 3500$ K.
\item 
    Your equation from (\ref{item:dlnrdlnm}) should rearrange to
\begin{equation}
    \frac{d \ln R}{d \ln M} = \left[2 - \left(\frac{L}{\dot{M}}+\psi\right)\frac{10-2n}{3}\frac{R}{GM} \right]
\end{equation}
where $L$ comes from (\ref{item:luminosity}).

    Numerically integrate this equation and plot the radius as a function of
    mass for $\dot{M} = 10^{-5}$ $\msun$ yr$^{-1}$ and $f_{\rm acc}=3/4$. As an
    initial condition, use $R=2.5$ $\rsun$ and $M=0.01$ $\msun$, and stop the
    integration at a mass of $M=1.0$ $\msun$.
    (hint: use \texttt{scipy.solve\_ivp} with \texttt{method='LSODA'})

    Plot the radius and luminosity as
    a function of mass; in the luminosity, include both the the accretion
    luminosity and the internal luminosity produced by the star.
\item Now explore what happens if you change the initial conditions.
    Create new plots of the radius and luminosity as a function of mass, but now
    use $R_0=0.5 R_\odot$. 
\item 
    Create new plots of the radius and luminosity as a function of mass, but now
    use $\dot{M}=10^{-6} M_\odot \mathrm{yr}^{-1}$. 
\item Now consider two modifications we can make to allow the model to work
for massive protostars.

First, since massive stars are radiative, the
polytropic index will be roughly $n=3$ rather than $n=3/2$. Second, the
surface temperature will in general be larger than the Hayashi limit, so take
the luminosity to be $L=\max[L_{\rm H}, \lsun(M/\msun)^3]$, where $L_{\rm
H}=4\pi R^2 \sigma T_{\rm H}^4$ and $R$ is the stellar radius.

Modify your
evolution equation for the radius to include these effects, and numerically
integrate the modified equations up to $M=50$ $\msun$ for $\dot{M} = 10^{-4}$
$\msun$ yr$^{-1}$ and $f_{\rm acc}=3/4$, using the same initial conditions as
for the low mass case. Plot $R$ and $L$ versus $M$.
%\item Compare your result to the fitting formula for the ZAMS radius of solar-metallicity stars as a function of $M$ in \citet{tout96a}\footnote{\href{http://adsabs.harvard.edu/abs/1996MNRAS.281..257T}{Tout et al., 1996, MNRAS 281, 257}}. Find the mass at which the massive star would join the main sequence. Your plots for $R$ and $L$ are only valid up to this mass, because this simple model does not include hydrogen burning.
\end{enumerate}



%     \item Problem: Opacities.
% 
%         You have seen that opacity is important for a wide range of physical processes, since it governs how light interacts with matter.
%         \begin{enumerate}
%             \item Using Karl Gordon's Dust Extinction package (https://dust-extinction.readthedocs.io/en/stable/), determine what the opacity
%                 of dust is at 1000 angstroms and 10000 angstroms.  How much more opaque is dust to UV than to IR light?
%                 (use the CCM89 model from \url{https://dust-extinction.readthedocs.io/en/stable/dust_extinction/choose_model.html#average-models})
%             \item Planck mean
%         \end{enumerate}

%     \item Stellar evolution
% 
%         
% 2. Determine the equation for the luminosity time evolution of a pre-main-sequence
% star. The equation should be in terms of the initial luminosity and μ at t=0,
% the mass of the star, and the energy generation per gram of gas. First
% calculate the rate at which Hydrogen is converted into Helium and from this get
% dX/dt (assume the star is just Helium and Hydrogen). From this measure the time
% variation of μ. hen use the variation of μ with time to get the change in
% luminosity.


\end{enumerate}

\end{document}
